\documentclass{article}

\usepackage[utf8]{inputenc}
\usepackage[french]{babel}
\usepackage{fullpage}
\usepackage{amsmath}
\usepackage{amssymb}

\title{Formulaire de mathématiques}
\author{Titouan \bsc{Christophe}}
\date{\today}

\begin{document}
\maketitle

\section{Séries}
\subsection{Opérations}
\begin{equation}
    \sum_{i=1}^{n} ai = a\sum_{i=1}^{n} i
\end{equation}

\begin{equation}
    \sum_{i=1}^{n} (i^x + i^y) = \sum_{i=1}^{n} i^x + \sum_{i=1}^{n} i^y
\end{equation}

\subsection{Valeur}
\begin{equation}
    \sum_{i=1}^{n} i = \frac{n(n+1)}{2}
\end{equation}

\begin{equation}
    \sum_{i=k}^{n} (ai + b) = a\times\frac{n(n+1) - k(k-1)}{2} + b(n-k+1)
\end{equation}

\begin{equation}
    \sum_{i=1}^{n} i^2 = \frac{n(n+1)(2n+1)}{6}
\end{equation}

\subsection{Convergence}
Si $|x| < 1$ 
\begin{equation}
    \sum_{i=1}^{n} x^i = \frac{1}{1-x}
\end{equation}

\section{Analyse de fonctions}
\subsection{Fonctions réelles}
Soit $ f : \mathbb{R}^2 \rightarrow \mathbb{R} : (x, y) \rightarrow f(x, y) = z $. 
Sa matrice Hessienne est donnée par
\begin{equation}
    \mathbb{H}_f = 
    \begin{pmatrix}
       \frac{\partial^2f}{\partial x^2} & \frac{\partial^2f}{\partial y\partial x} \\
       \frac{\partial^2f}{\partial x\partial y} & \frac{\partial^2f}{\partial y^2} 
    \end{pmatrix}
    =
    \begin{pmatrix}
        r(x,y) & s(x,y) \\
        s(x,y) & t(x,y)
    \end{pmatrix}
\end{equation}

\begin{itemize}
    \item Si $det(\mathbb{H}_f)(x, y) < 0$, alors $(x, y)$ est un point de selle
    \item Si $r(x, y) > 0$, alors $(x, y)$ est un minimum local
    \item Si $r(x, y) < 0$, alors $(x, y)$ est un maximum local
\end{itemize}

\end{document}
