\documentclass{article}

\usepackage[utf8]{inputenc}
\usepackage[french]{babel}
\usepackage{fullpage}
\usepackage{amsmath}
\usepackage{amssymb}

\title{Formulaire de mathématiques}
\author{Titouan \bsc{Christophe}}
\date{\today}

\begin{document}
\maketitle

\section{Suites}
Si la suite $|a_k|_k$ converge, alors la suite $(a_k)_k$ converge également, mais la réciproque n'est pas vraie
\begin{equation}
    \lim_{k\rightarrow\inf}(1+\frac{1}{k})^{xk} = e^x
\end{equation}


\section{Séries}
\subsection{Opérations}
\begin{equation}
    \sum_{i=1}^{n} \lambda i = \lambda\sum_{i=1}^{n} i
\end{equation}

\begin{equation}
    \sum_{i=1}^{n} (i^x + i^y) = \sum_{i=1}^{n} i^x + \sum_{i=1}^{n} i^y
\end{equation}

\subsection{Valeur}
\begin{equation}
    \sum_{i=1}^{n} i = \frac{n(n+1)}{2}
\end{equation}

\begin{equation}
    \sum_{i=k}^{n} (ai + b) = a\times\frac{n(n+1) - k(k-1)}{2} + b(n-k+1)
\end{equation}

\begin{equation}
    \sum_{i=1}^{n} i^2 = \frac{n(n+1)(2n+1)}{6}
\end{equation}

\subsection{Convergence}
Si $|x| < 1$ 
\begin{equation}
    \sum_{i=1}^{\inf} x^i = \frac{1}{1-x}
\end{equation}

\paragraph{Série harmonique}
\begin{equation}
    \sum_{i=1}^{\inf} \frac{1}{i^m}
\end{equation}
Converge ssi $ m > 1 $

\section{Analyse de fonctions}
\subsection{Fonctions réelles}
Soit $ f : \mathbb{R}^2 \rightarrow \mathbb{R} : (x, y) \rightarrow f(x, y) = z $. 
Sa matrice Hessienne est donnée par
\begin{equation}
    \mathbb{H}_f = 
    \begin{pmatrix}
       \frac{\partial^2f}{\partial x^2} & \frac{\partial^2f}{\partial y\partial x} \\
       \frac{\partial^2f}{\partial x\partial y} & \frac{\partial^2f}{\partial y^2} 
    \end{pmatrix}
    =
    \begin{pmatrix}
        r(x,y) & s(x,y) \\
        s(x,y) & t(x,y)
    \end{pmatrix}
\end{equation}

\begin{itemize}
    \item Si $det(\mathbb{H}_f)(x, y) < 0$, alors $(x, y)$ est un point de selle
    \item Si $r(x, y) > 0$, alors $(x, y)$ est un \textbf{minimum} local
    \item Si $r(x, y) < 0$, alors $(x, y)$ est un \textbf{maximum} local
\end{itemize}


\section{Série de Fourier réelle}
Soit $ f : \mathbb{R} \rightarrow \mathbb{R} $, $2\pi$ périodique. Sa série de Fourier est
\begin{equation}
    \begin{cases}
        S_f(t) = \frac{a_0}{2} + \sum_{n=1}^{\inf} a_n \cos(nt) + b_n \sin(nt)\\
        a_0 = \frac{1}{\pi}\int_{-\pi}^{\pi}f(t)dt\\
        a_n = \frac{1}{\pi}\int_{-\pi}^{\pi}f(t)\cos(nt)dt\\
        b_n = \frac{1}{\pi}\int_{-\pi}^{\pi}f(t)\sin(nt)dt
    \end{cases}
\end{equation}

\begin{itemize}
    \item Si $f$ est \textbf{paire}, tous les coefficients $b_n$ sont nuls.
    \item Si $f$ est \textbf{impaire}, tous les coefficients $a_n$ sont nuls.
\end{itemize}

\section{Quelques dérivées et intégrales utiles}
\begin{equation}
    \frac{\partial f}{\partial x} \sin(kx) = k \cos(kx) 
    \Leftrightarrow 
    \int \cos(kx) dx = \frac{\sin(kx)}{k}
\end{equation}

\begin{equation}
    \frac{\partial f}{\partial x} k \sin(x) = k \cos(x) 
    \Leftrightarrow 
    \int k \cos(x) dx = k \sin(x)
\end{equation}

\begin{equation}
    \frac{\partial f}{\partial x} \sin^2(x) = 2\sin(x)\cos(x) 
\end{equation}

\begin{equation}
    \int r \sin^2(\theta) d\theta = \frac{r^2[2\theta - \sin(2\theta)]}{4}
\end{equation}
\end{document}
